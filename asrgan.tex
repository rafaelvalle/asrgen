\documentclass{article}

% if you need to pass options to natbib, use, e.g.:
% \PassOptionsToPackage{numbers, compress}{natbib}
% before loading nips_2017
%
% to avoid loading the natbib package, add option nonatbib:
% \usepackage[nonatbib]{nips_2017}

\usepackage{nips_2017}

% to compile a camera-ready version, add the [final] option, e.g.:
% \usepackage[final]{nips_2017}

\usepackage[utf8]{inputenc} % allow utf-8 input
\usepackage[T1]{fontenc}    % use 8-bit T1 fonts
\usepackage{hyperref}       % hyperlinks
\usepackage{url}            % simple URL typesetting
\usepackage{booktabs}       % professional-quality tables
\usepackage{amsfonts}       % blackboard math symbols
\usepackage{nicefrac}       % compact symbols for 1/2, etc.
\usepackage{microtype}      % microtypography
% set imports
\usepackage{ntheorem}       % theorem writing
\usepackage{cleveref}       % clever references
\usepackage{graphicx}       % for images
\usepackage[table,xcdraw]{xcolor} % for tables
\usepackage{natbib}            % bibliography
\usepackage{amsmath, amssymb}  % math
\usepackage{bm}                % ?
\usepackage{subcaption}        % images
\usepackage{hyperref}          % linked refs

% set paths
\graphicspath{{./fig/}}

\title{Attacking Speaker Recognition with \\Deep Generative Models}
% The \author macro works with any number of authors. There are two
% commands used to separate the names and addresses of multiple
% authors: \And and \AND.
%
% Using \And between authors leaves it to LaTeX to determine where to
% break the lines. Using \AND forces a line break at that point. So,
% if LaTeX puts 3 of 4 authors names on the first line, and the last
% on the second line, try using \AND instead of \And before the third
% author name.

\author{
  Anish Doshi \\
  UC Berkeley \\
  Berkeley, 94709 \\
  \texttt{apdoshi@berkeley.edu} \\
  \And
  Wilson Cai\\
  UC Berkeley\\
  \texttt{wcai@berkeley.edu} \\
  \And
  Rafael Valle \\
  Center for New Music and Audio Technologies \\
  UC Berkeley\\
  \texttt{rafaelvalle@berkeley.edu} \\
}

\begin{document}
% \nipsfinalcopy is no longer used

\maketitle

\begin{abstract}
    In this paper we investigate targeted and untargeted attacks on speaker
    recognition systems. We first investigate the efficiency of SampleRNN 
    and Wavenet in fooling GMM-UBM and CNN speaker recognizers. We design
    untargeted and targetted attacks based on the GAN framework. We propose  
    a modification of the WGAN objective function to make use of data that is 
    real but not from the class being learned. Our method if efficient in 
    performing targetted and untargeted attacks, thus raising attention to
    issues related with security. 
\end{abstract}

% setup theorems 
\theoremseparator{:}
\newtheorem{hyp}{Hypothesis}

\section{Introduction} \label{sec:introduction}
Speaker authentication systems are being deployed for security critical
applications in industries like banking, forensics, and home automation. Like
other domains, such industries have benefited from recent advancements in deep
learning that lead to improved accuracy and trainability of the speech
authentication systems.  Despite the improvement in the efficiency of these
systems, evidence shows that they can be susceptible to adversarial
attacks\cite{wu2015spoofing}, thus motivating a current focus on understanding
adversarial attacks (\cite{szegedy2013intriguing},
\cite{goodfellow2014explaining}), finding countermeasures to detect and deflect
them and designing systems that are provably correct with respect to
mathematically-specified requirements~\cite{seshia2016vai}.

Parallel to advancements in speech authentication, neural speech
\textit{generation} (the process of using deep neural networks to generate
speech) has also seen huge progress in recent years \cite{wang2017tacotron}.  The combination of these advancements begs a natural
question that has, to the best of our knowledge, not yet been answered:
\begin{center}
Are speech authentication systems robust \\to adversarial attacks by speech generative models?
\end{center}

Generative Adversarial Networks (GANs) are generative models that recently have
been used to produce incredibly authentic samples in a variety of fields. The
core idea of GANs, a minimax game played between a generator network and a
discriminator network, extends naturally to the field of speaker authentication
and spoofing. 

With regards to this question, we offer in this research the following contributions:
\begin{itemize}
\item We evaluate samples produced with SampleRNN and WaveNet in their ability to fool text-independent speaker recognizers.
\item We propose strategies for untargeted attacks using Generative Adversarial Networks.
\item We propose a semi-supervised approach for targeted attacks by modifying
    Wasserstein's GAN loss function.
\end{itemize}

% The main evaluation metric for deep models has always been their qualitative ability to produce human-sounding speech. In this research, we will argue that using speech authenticators as validation

%
\section{Related work}\label{sec:related_work}
Generative models for speech can produce fake\footnote{We use the term fake to refer to
computer generated samples} samples that look very similar to real samples,
leading humans to believe that the fake samples are real.
WaveNet~\cite{van2016wavenet} is a generative neural network trained end-to-end
to model quantized audio waveforms that has produced impressive results for
generation of speech audio conditioned on speaker and text. The model is fully
probabilistic and autoregressive, using a stack of causal convolutional layers
to condition the predictive distribution for each audio sample on all previous ones; 

SampleRNN~\cite{mehri2016samplernn}, is another autoregressive architecture that
has been successfully used to generate both speech and music samples. SampleRNN
uses a hierarchical structure of deep RNNs to model dependencies in the sample
sequence. Each deep RNN operates at a different temporal resolution so as to
model both long term and short term dependencies. Another impressive model is
Adobe's VoCo~\cite{adobe2017voco}. Although VoCo's research is unpublished,
Adobe its hability to \textit{edit a recorded audio clip via text and make
a audio clip of someone saying sentence that he never said!}. Training the model
requires as litle as 20 minutes of speech.

Another interesting framework for generative models is the Generative Adversarial
Networks (GAN) framework proposed by~\cite{goodfellow2014generative}, in which a
\textit{generator} network is trained to learn a function from noise to samples
that approximate the real data distribution. Simultaneously, a
\textit{discriminator} network is trained to identify whether a sample came from
the real distribution or not - i.e., it is trained to try to output 1 if a sample is real, and 0 if a sample is fake. The generator and discriminator can be arbitrary networks.

The GAN framework has been shown to be able to produce very realistic samples with low training overhead. However, since the generator is trained to minimize the Kullback-Leibler (KL) divergence between its constructed distribution and the real one, it suffers from an exploding loss term when the real distribution's support isn't contained in the constructed one. To counter this, the \textit{Wasserstein GAN} \cite{arjovsky2017wasserstein} (WGAN) framework instead uses the Wasserstein (Earth-Mover) distance between distributions instead, which in many cases does not suffer from the same explosion of loss and gradient. Based on this, the loss functions of the generator and \textit{critic} (which no longer emits a simple probability, but rather an approximation of the Wasserstein distance between the fake distribution and real) become:
\begin{align}
    L_G &= -\underset{\boldsymbol{\widetilde{x}} \sim \mathbb{P}_{g}}{\mathbb{E}}  \big[D(\boldsymbol{\widetilde{x}})\big] \\
    L_C &= \underset{\boldsymbol{\widetilde{x}} \sim \mathbb{P}_{g}}{\mathbb{E}}  \big[D(\boldsymbol{\widetilde{x}})\big] - \underset{\boldsymbol{x} \sim \mathbb{P}_{r}}{\mathbb{E}}  \big[D(\boldsymbol{x})\big]
\end{align}
where $P_r$ is the real distribution, and $P_g$ the learnt distribution of the generator. \\
The original WGAN framework uses weight clipping to ensure that the critic satisfies a Lipschitz condition. As pointed by \cite{gulrajani2017improved}, however, this clipping can lead to problems with gradient stability. Instead, \cite{gulrajani2017improved} suggest adding a gradient penalty to the critic's loss function, which indirectly tries to constrain the original critic's gradient to have norm close to 1. Equation (2) thus becomes (taken from \cite{gulrajani2017improved}):
\begin{align}
    L_C &= \underbrace{\underset{\boldsymbol{\widetilde{x}} \sim \mathbb{P}_{g}}{\mathbb{E}}  \big[D(\boldsymbol{\widetilde{x}})\big] - \underset{\boldsymbol{x} \sim \mathbb{P}_{r}}{\mathbb{E}}  \big[D(\boldsymbol{x})\big]}_\text{Original critic loss}  + \underbrace{\lambda \underset{\boldsymbol{\hat{x}} \sim \mathbb{P}_{\hat{x}}}{\mathbb{E}}  \big[(\lVert \nabla_{\boldsymbol{\hat{x}}} D(\boldsymbol{\hat{x}}) \rVert_2 - 1)^2\big]}_\text{Gradient Penalty}
\end{align}

In the context of deep learning architectures as the ones descriebd in this
section, adversarial examples can make small perturbations to the original inputs, 
normally imperceptible to humans, to obtain an incorrect, target or untargeted, 
output from the neural network. In their brilliant papers, ~\cite{szegedy2013intriguing} and
~\cite{goodfellow2014explaining} analyze the origin of adversarial attacks and
describe simple and very efficient techniques for adversarial attacks, such as the fast gradient sign method. 

In the vision domain, ~\cite{sharif2016accessorize} describe a technique for attacking facial recognition systems. Their attacks are physically realizable and inconspicuous, allowing an attacker to impersonate another individual. In the speech domain,~\cite{carlini2016hidden} describe attacks on speech-recognition systems that use sounds that are hard to recognize by humans but interpreted as specific commands by speech-recognition systems.

%
\section{Method}\label{sec:method}
In this section we will describe the datasets used and the pipeline, including pre-processing and feature extraction, 

\subsection{Datasets}
In our experiments we use three datasets, each assigned to a model as described in Table\ref{tbl:datasets}. The datasets used are public and provide audio clips of different lengths and quality.

\begin{table}[!h]
\centering
\caption{Description of datasets used in our experiments. Book narr. refers to book narratives. Newspaper ++ refers to newspapers and other documents. Conversational tel. refers to conversational telephone speech.}
\label{tbl:datasets}
\begin{tabular}{llllll}
                                                                     & \cellcolor[HTML]{C0C0C0}Speakers & \cellcolor[HTML]{C0C0C0}Language & \cellcolor[HTML]{C0C0C0}Duration & \cellcolor[HTML]{C0C0C0}Context & \cellcolor[HTML]{C0C0C0}Model \\ \cline{2-6} 
\multicolumn{1}{l|}{\cellcolor[HTML]{C0C0C0}2013 Blizzard} & 1                                & English                          & 73 h                             & Book narr.                  & SampleRNN                     \\
\multicolumn{1}{l|}{\cellcolor[HTML]{C0C0C0}CSTR VCTK}               & 109                              & English                          & 400 Sentences                    & Newspaper ++               & WaveNet                       \\
\multicolumn{1}{l|}{\cellcolor[HTML]{C0C0C0}2004 NIST}               & 100                              & Multiple                         & 5 min / speaker                  & Conversational tel. & WGAN                         
\end{tabular}
\end{table}

\subsection{Pre-processing}
\label{sub:processdata}
Data pre-processing is dependent on the model being trained. For SampleRNN and WaveNet, the raw audio is reduced to 16kHz and quantized using the $\mu-law$ companding transformation as referenced in SampleRNN~\cite{mehri2016samplernn} and WaveNet~\cite{van2016wavenet}. For the model based on the Wasserstein GAN, we pre-process the data by converting it to 16kHz and removing silences by using the WebRTC Voice Activity Detector (VAD) as referenced in~\cite{zeidan2014webrtc}.

\subsection{Feature extraction}
SampleRNN and WaveNet operate at the sample level, i.e. waveform, thus requiring no feature extraction.
The features used for the neural speaker recognition system is based on Mel-Spectrograms with dynamic range compression. The Mel-Spectrogram is obtained by projecting a spectrogram onto a mel scale. We use the python library librosa~\cite{mcfee2015librosa} to project the spectrogram onto 64 mel bands, with window size equal to 1024 samples and hop size equal to 160 samples, i.e. frames of 100ms long. Dynamic range compression is computed as described in~\cite{lukic2016speaker}, with $log(1 + C*M)$, where $C$ is the a compression constant scalar set to $1000$ and $M$ is the matrix representing the Mel-Spectrogram.
                        
\section{Classification Model}
\subsection{Gaussian Mixture Model - Universal Background Model}
\subsection{Neural speaker recognition system}
The speaker recognition system used in our experiments is based on the state-of-the-art framework by \cite{lukic2016speaker} and is described in Figure \ref{fig:CNN}. The first module at the bottom is a pre-processing step that extracts Mel-Spectrograms features from the waveform as described in section \ref{sub:processdata}. The second module is a convolutional neural network (CNN) that performs multi-speaker classification using the Mel-Spectrograms. The CNN is a modified version of Alexnet~\cite{krizhevsky2012imagenet}.

\begin{figure}[h]
    \centering
    \includegraphics[width=0.25\textwidth]{./fig/cnn.png}
    \caption{Architecture for CNN speaker verifier}
    \label{fig:CNN}
\end{figure}

We train the CNN on our training set using 64*64 Mel-Spectrograms~\footnote{64 mel bands and 64 frames, 100 ms each} consisting of balanced samples from 101 speakers from the NIST 2004 and Blizzard datasets. Our model achieves 85\% test set accuracy.

\section{Generative Model}
\subsection{WaveNet} % (fold)
\label{sub:WaveNet}
WaveNet is a generative neural network trained end-to-end to model quantized audio waveforms. It has produced impressive results for conditional, speaker and text, generation of speech audio. The model is fully probabilistic and autoregressive, using a stack of causal convolutional layers to condition the predictive distribution for each audio sample on all previous ones;
\subsection{SampleRNN}
SampleRNN \cite{mehri2016samplernn}, another autoregressive architecture that has been successfully used to generate both speech and music samples. SampleRNN uses a hierarchical structure of deep RNNs to model dependencies in the sample sequence. Each deep RNN operates at a different temporal resolution so as to model both long term and short term dependencies.  
\subsection{Wasserstein GANs} % (fold)
\label{sub:Wasserstein_GANsh}
In the original generative adversarial network (GAN) framework proposed by \cite{goodfellow2014generative}, a \textit{generator} network is trained to learn a function from noise to samples that approximate the real data distribution. Simultaneously, a \textit{discriminator} network is trained to identify whether a sample came from the real distribution or not - i.e., it is trained to try to output 1 if a sample is real, and 0 if a sample is fake. The generator and discriminator can be arbitrary networks. \\
The GAN framework has been shown to be able to produce very realistic samples with low training overhead. However, since the generator is trained to minimize the Kullback-Leibler (KL) divergence between its constructed distribution and the real one, it suffers from an exploding loss term when the real distribution's support isn't contained in the constructed one. To counter this, the \textit{Wasserstein GAN} \cite{arjovsky2017wasserstein} (WGAN) framework instead uses the Wasserstein (Earth-Mover) distance between distributions instead, which in many cases does not suffer from the same explosion of loss and gradient. Based on this, the loss functions of the generator and \textit{critic} (which no longer emits a simple probability, but rather an approximation of the Wasserstein distance between the fake distribution and real) become:
\begin{align}
    L_G &= -\underset{\boldsymbol{\widetilde{x}} \sim \mathbb{P}_{g}}{\mathbb{E}}  \big[D(\boldsymbol{\widetilde{x}})\big] \\
    L_C &= \underset{\boldsymbol{\widetilde{x}} \sim \mathbb{P}_{g}}{\mathbb{E}}  \big[D(\boldsymbol{\widetilde{x}})\big] - \underset{\boldsymbol{x} \sim \mathbb{P}_{r}}{\mathbb{E}}  \big[D(\boldsymbol{x})\big]
\end{align}
where $P_r$ is the real distribution, and $P_g$ the learnt distribution of the generator. \\
The original WGAN framework uses weight clipping to ensure that the critic satisfies a Lipschitz condition. As pointed by \cite{gulrajani2017improved}, however, this clipping can lead to problems with gradient stability. Instead, \cite{gulrajani2017improved} suggest adding a gradient penalty to the critic's loss function, which indirectly tries to constrain the original critic's gradient to have norm close to 1. Equation (2) thus becomes (taken from \cite{gulrajani2017improved}):
\begin{align}
    L_C &= \underbrace{\underset{\boldsymbol{\widetilde{x}} \sim \mathbb{P}_{g}}{\mathbb{E}}  \big[D(\boldsymbol{\widetilde{x}})\big] - \underset{\boldsymbol{x} \sim \mathbb{P}_{r}}{\mathbb{E}}  \big[D(\boldsymbol{x})\big]}_\text{Original critic loss}  + \underbrace{\lambda \underset{\boldsymbol{\hat{x}} \sim \mathbb{P}_{\hat{x}}}{\mathbb{E}}  \big[(\lVert \nabla_{\boldsymbol{\hat{x}}} D(\boldsymbol{\hat{x}}) \rVert_2 - 1)^2\big]}_\text{Gradient Penalty}
\end{align}

% subsection wgan_approach (end)
\subsection{Adversarial attacks}
Attacks to classification systems can be targeted or untargeted. In targeted attacks, an adversary is interested in producing an adversarial input that makes the classification system predict a target class. In untargeted attacks, the adversary is interest in a confident prediction, regardless of the class.

%
\section{Results on existing methods}\label{sec:res_existing}
This section investigates the output of speaker recognition systems when
features computed from fake data generated with WaveNet and sampleRNN is used as
input. As described in \ref{sub:speaker_recognition}, we train a speaker
recognition systems with several speakers, including the speakers used to train
WaveNet and sampleRNN. 

\subsection{WaveNet} % (fold)
\label{sub:res-WaveNet}
We attempted to replicate the model described in~\cite{van2016wavenet} but,
unfortunately, we do not have access to Google's computing power nor to the
North American Speech dataset Google used to train the WaveNet model that
produced the samples referenced in \cite{van2016wavenet}. Nonetheless, we used
the data from CSTR VCTK to train speaker dependent WaveNet speech synthesis
models that converged to producing speech that resembled the speakers voice but
sounded like babbling. None of the WaveNet samples produced with our model were
successful in targeted or untargetted attacks to the speaker recognition
system~\footnote{Neither was it successful attacking a GMM-UBM speaker
recognizer.\label{foot:gmm_ubm}.}
% subsection WaveNet (end)
\subsection{sampleRNN} % (fold)
\label{sub:samplernn}
We also tested samples from sampleRNN \cite{mehri2016samplernn}, another
autoregressive architecture that has been successfully used to generate both
speech and music samples. SampleRNN uses a hierarchical structure of deep RNNs
to model dependencies in the sample sequence. Each deep RNN operates at a
different temporal resolution so as to model both long term and short term
dependencies. We generated samples from the three-tiered variant, trained on the
Blizzard 2013 dataset \cite{prahallad2013blizzard}, a 300 hour corpus of a
single female speaker's narration. We also downloaded 10 second samples from the
original paper's online repository at
\texttt{https://soundcloud.com/samplernn/sets}, which we qualitatively found to
have less noise than our generated ones. None of the sampleRNN samples produced
with our model nor with the model produced by~\cite{mehri2016samplernn} were
successful in targeted or untargetted attacks to the speaker recognition
system.~\ref{foot:gmm_ubm}
% subsection SampleRNN (end)

%
\section{Results on adversarial methods}\label{sec:res_adversarial}
\subsection{GAN Mel-Spectrogram}
Using the improved Wasserstein GANs framework, we trained generators to
construct 64x64 Mel-Spectrogram images from a noise vector. We used two popular
architectures for generator/discriminator pairs: 
\begin{itemize}
    \item \textit{DCGAN}~\cite{radford2015unsupervised} models the generator as a series of deconvolutional layers with ReLU activations, and the discriminator as a series of convolutional ones with leaky ReLU activations. Both architectures use batch normalization after each layer.
    \item \textit{ResNet}~\cite{ledig2016photo} models the generator and discriminator each as very deep convnets (30 layers in our experiments) with upsampling/downsampling respectively. Residual (skip) connections are added every few layers to make training easier.
\end{itemize}
Visual results are demonstrated below in Figure~\ref{fig:samples_comparison}. We
generally~\footnote{We use different parameters and loss function for targeted
attacks.} used the same parameters as \cite{gulrajani2017improved}, namely 5
critic iterations per generator iteration, a gradient penalty weight of 10, 
and batch size of 64. We saw recognizable Mel-Spectrogram-like features in the
data after only 1000 generator iterations, and after 5000 iterations the
generated samples were indistinguishable from real ones. Training took around 10
hours for 20000 iterations on a single 4 GB Nvidia GK104GL GPU.
\begin{figure}[h]
    \centering
    \begin{subfigure}[b]{0.4\textwidth}
        \includegraphics[width=\textwidth]{./fig/samples_groundtruth.png}
        \caption{Real (actual)}
        \label{fig:samples_real}
    \end{subfigure}
    \qquad
    \begin{subfigure}[b]{0.4\textwidth}
        \includegraphics[width=\textwidth]{./fig/samples_5419.png}
        \caption{Fake (generated)}
        \label{fig:samples_fake}
    \end{subfigure}
    \caption{Comparison of real and generated ($\sim$ 5000 generator iterations)
    spectrogram samples from all speakers. Each grid contains 64 samples.}
    \label{fig:samples_comparison}
\end{figure}

\subsection{GAN Adversarial attacks}

Within the GAN framework, we train models for untargeted attacks by using all
data available from speakers that the speaker recognition systems was trained on, 
irrespective of class label. We show that an untargeted model able to generate 
data from the real distribution with enough variety can be used to perform 
adversarial attacks. We provide details in the untargeted attacks 
subsection \ref{sub:untargeted}. Figure~\ref{fig:histogram_untargeted} depicts
that our GAN-trained generator successfully learns all speakers across the
dataset, without mode collapsing.
\begin{figure}[t]
    \centering
    \begin{subfigure}[b]{0.4\textwidth}
        \includegraphics[width=\textwidth]{./fig/conf_mat_untargeted.png}
        \caption{Our speaker classifier's softmax distribution of 1000 samples 
        on approximately 100 speakers.}
        \label{fig:cm_untargeted}
    \end{subfigure}
    \qquad
    \begin{subfigure}[b]{0.4\textwidth}
        \includegraphics[width=\textwidth]{./fig/histogram_untargeted.png}
        \caption{Our speaker classifier's distribution of randomly sampled 
        speech from the generative model.}
        \label{fig:histogram_untargeted}
    \end{subfigure}
    \caption{Summary of untargeted attacks. Red represents high confidence.}
    \label{fig:histogram_summary}
\end{figure}

The models for targeted attacks can be trained in two manners: 1) 
conditioning the model on additional information, e.g. class labels, as
described in~\cite{mirza2014conditional}; 2) using only data from the label 
of interest. While the first approach might result in mode collapse, a drawback
of the second approach is that the discriminator, and by consequence the
generator, does not have access to universal\footnote{We draw a parallel with 
Universal Background Models in speech.}. properties of speech. In the targeted 
attacks subsection \ref{sub:targeted} we propose a new objective function that allows 
using the data from all speakers.  

\subsubsection{Untargeted attacks}
\label{sub:untargeted}
For each speaker audio data in the test set, we compute a Mel-Spectrogram as
descibred in section \ref{sub:processdata}. The resulting Mel-Spectrogram is
then fed into the CNN recognizer and we extract a 505-dimensional feature $G$ from
the penultimate fully-connected layer (L7) in the pre-trained CNN model
(\ref{fig:CNN}) trained on the train partition of the real speech dataset with all 
speaker IDs.  This deep feature/embedding $G$ is then used to train a 
K-nearest-neighbor (KNN) classifier, with K equal to 5.

To control the generator trained by our WGAN, we feed the generated
Mel-Spectrograms into the same CNN-L2 pipeline to extract their corresponding
feature $\widehat G$. Utilizing the pre-trained KNN, each sample is assigned to
the nearest speaker in the deep feature space. Therefore, we know which speaker
our generated sample belongs to when we attack our CNN recognizer. We evaluate our
controlled WGAN samples against the state-of-the-art CNN recognizer, and the
confusion matrix can be found in Figure \ref{fig:conf_mat_cnn_knn}. Although not
included in the figure, \textbf{neither WaveNet samples nor SampleRNN samples
were able to attack the recognition model in the same way.}~\footnote{Unlike our
method, neither were they successful in attacking a GMM-UBM speaker recognizer.}


\subsubsection{Targeted attacks}
\label{sub:targeted}
However, the most natural attack is one in which we train a GAN to directly fool
a speaker recognition system, i.e., to produce samples that the system
classifies as matching a target speaker with reasonable confidence. We attempted
using data from the target speaker only, but the generated samples did not fool
the speaker recognition systems. To circumvent this, we propose a 
modification to the critic's objective function that allows it to learn to 
differentiate between not only real samples and generated samples, but also between real speech samples from a target 
speaker and real speech samples from other speakers. We do this by adding a term 
to the critic's loss that encourages its discriminator to classify real speech 
samples from untargeted speakers as fake. The critic's loss $L_C$ becomes:
\begin{align}
    \underbrace{\underset{\boldsymbol{\widetilde{x}} \sim \mathbb{P}_{g}}{\mathbb{E}}  \big[D(\boldsymbol{\widetilde{x}})\big]}_\text{Generated Samples} \color{red} + \alpha * \underbrace{\underset{\boldsymbol{\dot{x}} \sim \mathbb{P}_{\dot{x}}}{\mathbb{E}}  \big[D(\boldsymbol{\dot{x}})\big]}_\text{Different Speakers} \color{black} - \underbrace{\underset{\boldsymbol{x} \sim \mathbb{P}_{r}}{\mathbb{E}}  \big[D(\boldsymbol{x})\big]}_\text{Real Speaker}  + \underbrace{\lambda \underset{\boldsymbol{\hat{x}} \sim \mathbb{P}_{\hat{x}}}{\mathbb{E}}  \big[(\lVert \nabla_{\boldsymbol{\hat{x}}} D(\boldsymbol{\hat{x}}) \rVert_2 - 1)^2\big]}_\text{Gradient Penalty}
\end{align}
where $P_{\hat{x}}$ is the distribution of samples from other speakers, and
$\alpha$ is a tunable scaling factor. In our experiments, we trained this GAN
with 1 target speaker and a set of over 100 'other' speakers. On each critic
iteration, we would feed it with a batch of samples from one target speaker, 
and a batch of data uniformly sampled from the other speakers. During training 
we used two approaches. The first one used an $\alpha$ of 0.1 (we found that not 
including this scaling factor led to serious overfitting and poor convergence of 
the GAN). The second one used an $\alpha$ of 1 combined with modifications to 
the network parameters, including different weight initialization, learning
rates and addition of gaussian noise to the target speaker data. We invite
readers to our github repo for details. Results are demonstrated in Figure
\ref{fig:confusion_matrices}: the histogram of predictions in Figure
~\ref{fig:pred_comp_spk0} shows that for the Mixed loss most of the energy is
concentrated on the target speaker 0. The improved WGAN loss achieves 0.39 error
rate and our mixed loss achieves 0.11 error rate, producing a 77\%
increase in accuracy. 
\begin{figure}[t]
    \centering
    \begin{subfigure}[b]{0.3\textwidth}
        \includegraphics[width=\textwidth]{./fig/conf_mat_cnn_knn.png}
        \caption{Confusion matrix of \\untargeted model}
        \label{fig:conf_mat_cnn_knn}
    \end{subfigure}
    \quad
    \begin{subfigure}[b]{0.4\textwidth}
        \includegraphics[width=\textwidth]{./fig/pred_comparisson_spk0.png}
        \caption{Histogram of predictions given improved WGAN and mixed loss models.}
        \label{fig:pred_comp_spk0}
    \end{subfigure}
    \caption{Confusion matrix of targeted attacks}
    \label{fig:confusion_matrices}
\end{figure}

%
\section{Discussion and Conclusion}\label{sec:conclusions}


In this paper we have investigated the use of speech generative models to perform adversarial attacks on speaker recognition systems. We show that the autoregressive models we trained, i.e. SampleRNN and WaveNet, were not able to fool the CNN speaker recognizers we built. On the other hand, we show that adversarial examples generated with GAN networks are successful in performing targeted and untargeted adversarial attacks.

A natural question to ask is whether existing speech synthesis architectures like WaveNet and SampleRNN can be augmented with an adversarial-type loss in the same way as GANs. Both WaveNet and SampleRNN are trained to minimize the cross entropy loss between their generated samples and the real data. If one could attach a term to this loss function in the same way (e.g., maximizing the l2 distance between the generated samples and the data from other speakers, and tuning the weight of this distance to allow convergence), perhaps such a modification could be made. This modification would valuable as well when considering more sophisticated architectures like \cite{wang2017tacotron}.
%A pertinent argument against the validity of the GMM-UBM tests lies on the fact
%that GMM-UBM models have high precision and would not generalize to speech in
%different conditions, e.g. different room and microphone conditions. First, it
%is not within the scope of this paper to build a speaker recognition system
%that is invariant to room and microphone conditions. Second, given that the
%speaker recognition models, GMM-UBM and CNN, have good performance on test data
%and that WaveNet and SampleRNN goal is to replicate speech data that is from a
%speaker with similar and fixed microphone and room conditions, it is expected
%that the outputs of these generative models should be properly classified by
%the speaker recognition system.%

With this paper we hope to raise attention to issues that generative models bring to security and biometric systems. We foresee that samples produced with generative models have a signature that can be used to identify the source of the data and leave this investigation for future work.

%

\subsubsection*{Acknowledgments}
\input{acknowledgments}

\section*{References}
\bibliographystyle{plain}
\bibliography{references.bib}

\end{document}
